Using the tokenized and annotated questions and answers from the MCTest dataset, we also created a system to automatically generate statements from question-answer pairs that would be useful for approaching the task as a recognizing textual entailment problem, to allow for feature extraction using the statements later in the system. We take a similar approach to the search engine query reformulation seen in \cite{brill02ananalysis}, but additionally utilise the benefit of having parsed text with part of speech tagging.

Our system uses multiple string-based manipulations to rewrite the question-answer pair into various different forms that could then be used as a hypothesis for evaluation. Rewrite rules are applied to the question-answer pair depending on the classification of the type of question, as well as the lexical structure of the answer.

Each possible rewrite is then evaluated using a language model trained using the SRILM toolkit \cite{Stolcke2002} with training data obtained from English Wikipedia. The system uses a 5-gram language model to assign a probability of occurrence to each possible rewrite and then the system chooses the rewrite with the highest probability to then be used for feature extraction in later parts of the system. 